% DOCUMENT CLASS
\documentclass[11pt, oneside]{article}
\usepackage{geometry}                	
\geometry{a4paper}

% SET SANS SERIF FONT
\renewcommand{\familydefault}{\sfdefault}
\renewcommand*\sfdefault{phv}

% FONTS PACKAGES
\usepackage{amsmath}
\usepackage{amsfonts}
\usepackage{amssymb}
\usepackage{courier}

% SKIP LINES BETWEEN PARAGRAPHS
\usepackage[parfill]{parskip}

% COLOR SETTINGS
\usepackage[usenames,dvipsnames]{xcolor}
% SET COLOR PALETTE (flattastic color palette BY erigon)
\definecolor{grapefruit}{HTML}{ED5565}
\definecolor{Grapefruit}{HTML}{DA4453}
\definecolor{bittersweet}{HTML}{FC6E51}
\definecolor{Bittersweet}{HTML}{E9573F}
\definecolor{sunflower}{HTML}{FFCE54}
\definecolor{Sunflower}{HTML}{F6BB42}
\definecolor{grass}{HTML}{A0D468}
\definecolor{Grass}{HTML}{8CC152}
\definecolor{mint}{HTML}{48CFAD}
\definecolor{Mint}{HTML}{37BC9B}
\definecolor{aqua}{HTML}{4FC1E9}
\definecolor{Aqua}{HTML}{3BAFDA}
\definecolor{jeans}{HTML}{5D9CEC}
\definecolor{Jeans}{HTML}{4A89DC}
\definecolor{lavender}{HTML}{AC92EC}
\definecolor{Lavender}{HTML}{967ADC}
\definecolor{rose}{HTML}{EC87C0}
\definecolor{Rose}{HTML}{D770AD}
\definecolor{lightgrey}{HTML}{F5F7FA}
\definecolor{LightGrey}{HTML}{E6E9ED}
\definecolor{mediumgrey}{HTML}{CCD1D9}
\definecolor{MediumGrey}{HTML}{AAB2BD}
\definecolor{darkgrey}{HTML}{656D78}
\definecolor{DarkGrey}{HTML}{434A54}

% BIBLIOGRAPHY STYLE
\bibliographystyle{ieeetr}

% LINKS SETTINGS
\usepackage{url}
\usepackage[colorlinks = true,
            linkcolor = black,
            urlcolor  = black,
            citecolor = black,
            anchorcolor = black]{hyperref}

% FOOTER HEIGH
\setlength{\footskip}{3cm}            		
\setlength{\skip\footins}{0.5cm}

% PACKAGES FOR PICTURES AND TABLES
\usepackage{booktabs}
\usepackage{graphicx} 
\usepackage{caption}
\usepackage{subcaption}
\usepackage{tabularx}

% TO USE ALL CARACTERS OF THE KEYBOARD
\usepackage[utf8]{inputenc} 
\usepackage[T1]{fontenc}

% FRENCH AND ENGLISH
\usepackage[francais, english]{babel}

% LISTING FOR CODING
\usepackage{listings}
\renewcommand{\lstlistingname}{Fichier}% Listing -> Algorithm
\renewcommand{\lstlistlistingname}{Liste des \lstlistingname s}% List of Listings -> List of Algorithms

\lstset{%
		tabsize=2,
		extendedchars=true,
		basicstyle=\footnotesize\ttfamily,
		breaklines=true,
		numbers=left, 
		numberstyle=\tiny, 
		stepnumber=1,
		backgroundcolor=\color{white},
		commentstyle=\color{MediumGrey},
		keywordstyle=\color{Jeans},
		stringstyle=\color{Sunflower},
		inputencoding=utf8,
            	extendedchars=true,
            	literate=%
            	{é}{{\'{e}}}1
            	{è}{{\`{e}}}1
            	{ê}{{\^{e}}}1
    	        {ë}{{\¨{e}}}1
        	{û}{{\^{u}}}1
        	{ù}{{\`{u}}}1
		{â}{{\^{a}}}1
		{à}{{\`{a}}}1
		{î}{{\^{i}}}1
		{ô}{{\^{o}}}1
		{ç}{{\c{c}}}1
		{Ç}{{\c{C}}}1
		{É}{{\'{E}}}1
		{Ê}{{\^{E}}}1
		{À}{{\`{A}}}1
		{Â}{{\^{A}}}1
		{Î}{{\^{I}}}1
}

% INFOS
\title{Extraction de motifs communs à plusieurs séquences biologiques}
\author{Victor Gaborit \& Clara Jégousse}
\date{}

% BEGIN DOCUMENT
\begin{document}

% LANGUE FRANCAISE
\selectlanguage{french}

\maketitle

% TABLE DES MATIERES
\setcounter{tocdepth}{3}
\tableofcontents
\clearpage

\part*{Rapport}

\section{Structures de données} 


% structures de données utilisées: schéma
\begin{figure}[h!]
\centering
  \includegraphics[width=\textwidth]{SchemaTTabSeq.png}
\caption{Schémas des structures de données}
\label{fig:structure_donnees_dictionnaire}
\end{figure}

\begin{figure}[h!]
\centering
  \includegraphics[width=\textwidth]{Schema.png}
\caption{Schémas des structures de données}
\label{fig:structure_donnees_dictionnaire}
\end{figure}
\newpage

\section{Tests}

% tests réalisés
%\lstinputlisting[language=, caption={structures.c}]{structures.c}

\section{Bilan}

% bilan concis récapitulant la réalisation effective par rapport au texte du devoir, et récapitulant ce qui reste à faire (par rapport au texte du devoir), ou au regard des améliorations que vous apporteriez si vous plus de temps.

Le programme rempli les critères suivant:
\begin{itemize}
\item lecture des séquences nucléiques à partir d'un fichier texte et mémorisation de ces séquences dans une liste dynamique;
\item création d'une structure de données "dictionnaire" pour stocker la liste des motifs;
\item demande en boucle d'un nom de fichier de séquences à l'utilisateur;
\item demande des paramètres : longueur du(des) motif(s) commun(s) à extraire, nombre maximum d'erreurs autorisé entre occurrence et motif, et valer du quorum;
\item affichage de la liste des motifs qui vérifient le quorum;
\item possibilité de consulter la liste des occurrences d'un motif sélectionné par l'utilisateur;
\item possibilité d'exporter le dictionnaire de motif dans un fichier texte.
\end{itemize}

\section{Limites et améliorations}  

% bilan concis des erreurs
\part*{Code source}

\lstinputlisting[language=C, caption={structures.c}]{structures.c}
\lstinputlisting[language=C, caption={Programme principal main.c}]{main.c}
\lstinputlisting[language=C, caption={fonctions.h}]{fonctions.h}
\lstinputlisting[language=C, caption={fonctions.c}]{fonctions.c}



\end{document}
